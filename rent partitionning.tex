\documentclass[12pt,twoside,a4paper]{article}
\usepackage{amsmath, amssymb, amsfonts, mathrsfs} %les plus utiles et quasi indispensables 
\usepackage{amsthm}%\usepackage[T1]{fontenc}
\usepackage{a4wide}
\usepackage[latin1]{inputenc}
\usepackage{fancyhdr} %pour les hauts et bas de page
\usepackage[francais]{babel}% donne de bonnes c\'esures  en francais
\usepackage{titlesec}%pour modifier le style des sections
\usepackage{float}%pour fixer les flottants
\usepackage{url}%pour ecrire une adresse d'un site web
\usepackage[all]{xy}%pour faire des diagrammes
\usepackage[dvips]{graphicx}
\usepackage{todonotes}
\usepackage{graphicx}
\usepackage{pgf,tikz}
\usetikzlibrary{arrows}
\usepackage{enumitem}
\usepackage[T1]{fontenc}
\usepackage{hyperref}
\usepackage{array}
\usepackage{chemfig}
\usepackage{graphics}
\usepackage{eurosym}
\usepackage{soul}
\usepackage{wasysym}
\usepackage{textcomp}
\usepackage{listings}
\usepackage{stmaryrd}



\title{\textbf{Synth\`ese d'article} :\\ Rental harmony : Sperner's lemma in fair division}
\author{\textsc{L\'eo Baty, Luca Brunod-Indrigo , Emmanuel Lo, Nicolas Podvin}}
\date{}


\begin{document}
\maketitle


\tableofcontents

\section{Introduction}

L'article \'etudi\'e\footnote{Rental harmony : Sperner's lemma in fair division, FRANCIS EDWARD SU} porte sur la r\'esolution du probl\`eme de \textit{rent-partitioning} \`a l'aide du lemme de Sperner. Le probl\`eme concret \`a r\'esoudre est le suivant : $n$ individus s'appr\^{e}tent \`a s'installer en collocation dans un logement poss\'edant $n$ chambres, et vient le moment de la r\'epartition des chambres. Si l'on associe \`a chaque chambre un pourcentage du loyer \`a payer, y a-t-il une r\'epartition du loyer optimale, telle que chaque colocataire pr\'ef\`ere une chambre diff\'erente ? 

\section{R\'esum\'e de l'article}

\subsection{Lemme de Sperner}

La triangulation d'un simplexe de dimension $n$ par des $n$-simplexes \'el\'ementaires est dite de Sperner lorsque les $n+1$ sommets sont index\'es par un chiffre diff\'erent et que pour tout sous-simplexe de dimension inf�rieure ayant uniquement des sommets dans l'ensemble des $n+1$ sommets du simplexe de base, toute triangulation de ce sous-simplexe par des simplexes \'el\'ementaires ne contiennent que des chiffres des sommets initiaux du sous-simplexe.

Une telle triangulation d'un simplexe poss\`ede une propri�t� d\'ecrite par le lemme de Sperner : un tel simplexe poss�de n\'ecessairement un nombre impair sous-simplexe �l�mentaire de m�me dimension n avec les $n+1$ diff�rentes indexations aux $n+1$ sommets de ce simplexe \'el\'ementaire
\subsection{R\'esolution du probl\`eme de \textit{rent-partitioning}}
%resumer le prb du gateau en premier?

Le lemme de Sperner trouve alors une application dans la vie courante : le probl\`eme de l'allocation des chambres entre plusieurs locataires. Pr\'esentons le probl\`eme : on a $n$ locataires qui payent un certain loyer pour occuper une chambre et doivent d\'ecider entre eux de la part de loyer \? a payer. Le but est de trouver une r\'epartition du loyer qui apporte un payement qui soit socialement juste pour tous les joueurs.
Les hypoth�ses prises sur les joueurs sont les suivantes :
\begin{enumerate}
	\item Tout locataire est satisfait par au moins une des chambres de la propri\'et\'e, quelque soit la r\'epartition du loyer.
	\item Une personne pr\'ef\`ere toujours une chambre gratuite \`a une chambre payante.
	\item Si une personne est satisfaite par une certaine chambre pour une suite convergente de r\'epartition de loyer, alors il est satisfait par la m\^{e}me chambre \`a la limite de cette r\'epartition.
\end{enumerate}
%expliquer en quoi cela se rapproche de Sperner
%parler de dualite?
%faire une fonction de payement theorie des jeux? ou pseudo fonction?

\subsection{Algorithme de r\'esolution}

\section{R\'eflexions personnelles}

\subsection{Adaptation du probl\`eme au cours}
%je ne suis pas sur de l'utilit\'e du truc si on fait le lien direct dans les parties

\subsection{Impl\'ementation de l'algorithme}

\subsection{Application \`a un cas r\'eel}


\end{document}