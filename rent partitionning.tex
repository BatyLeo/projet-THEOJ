\documentclass[12pt,twoside,a4paper]{article}
\usepackage{amsmath, amssymb, amsfonts, mathrsfs} %les plus utiles et quasi indispensables 
\usepackage{amsthm}%\usepackage[T1]{fontenc}
\usepackage{a4wide}
\usepackage[latin1]{inputenc}
\usepackage{fancyhdr} %pour les hauts et bas de page
\usepackage[francais]{babel}% donne de bonnes c\'esures  en francais
\usepackage{titlesec}%pour modifier le style des sections
\usepackage{float}%pour fixer les flottants
\usepackage{url}%pour ecrire une adresse d'un site web
\usepackage[all]{xy}%pour faire des diagrammes
\usepackage[dvips]{graphicx}
\usepackage{todonotes}
\usepackage{graphicx}
\usepackage{pgf,tikz}
\usetikzlibrary{arrows}
\usepackage{enumitem}
\usepackage[T1]{fontenc}
\usepackage{hyperref}
\usepackage{array}
\usepackage{chemfig}
\usepackage{graphics}
\usepackage{eurosym}
\usepackage{soul}
\usepackage{wasysym}
\usepackage{textcomp}
\usepackage{listings}
\usepackage{stmaryrd}



\title{\textbf{Synth\`ese d'article} :\\ Rental harmony : Sperner's lemma in fair division}
\author{\textsc{L\'eo Baty, Luca Brunod-Indrigo , Emmanuel Lo, Nicolas Podvin}}
\date{}


\begin{document}
\maketitle


\tableofcontents

\section{Introduction}

L'article \'etudi\'e\footnote{Rental harmony : Sperner's lemma in fair division, FRANCIS EDWARD SU} porte sur la r\'esolution du probl\`eme de \textit{rent-partitioning} \`a l'aide du lemme de Sperner. Le probl\`eme concret \`a r\'esoudre est le suivant : $n$ individus s'appr\^{e}tent \`a s'installer en collocation dans un logement poss\'edant $n$ chambres, et vient le moment de la r\'epartition des chambres. Si l'on associe \`a chaque chambre un pourcentage du loyer \`a payer, y a-t-il une r\'epartition du loyer optimale, telle que chaque colocataire pr\'ef\`ere une chambre diff\'erente ? 

\section{R\'esum\'e de l'article}

\subsection{Lemme de Sperner}

\subsection{R\'esolution du probl\`eme de \textit{rent-partitioning}}

\subsection{Algorithme de r\'esolution}

\section{R\'eflexions personnelles}

\subsection{Impl\'ementation de l'algorithme}

\subsection{Application \`a un cas r\'eel}


\end{document}